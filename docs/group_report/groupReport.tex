\documentclass[12pt,a4paper]{article}

\usepackage{epcc}
\usepackage{graphics}

% This example file shows how a thesis can be laid out using Latex. It
% does not use any special local features so should be portable to other
% places.
% 
% When producing draft copies of a thesis you may want to print only
% selected pages of the thesis. To do this use the command
% 
% dvips -f -p 4 -n 3 myfile.dvi | lpr
% 
% where -p 4 means start printing at page 4 (ie the page that will be
% numbered 4, not necessarily the 4th page) and -n 3 means print 3 pages.
% This example will print pages 4, 5 and 6.
% 
% If you want to print the thesis and also save paper you can print more
% than one page on each sheet of paper. Use the command
% 
% dvips -f myfile.dvi | psnup -2 | lpr
% 
% to print 2 pages per sheet. psnup can take values 2, 4, 8, or 9.
%
% To produce a PDF version you can create a PostScript copy first
%
% dvips -f myfile.dvi > myfile.pdf
%
% and then convert it
%
% distill myfile.ps
%
% or you can go straight to PDF
%
% pdflatex myfile
%
% Note that pdflatex expects all included figures to be in PDF too. See
% the includegraphics command below.


% This document contains many cross-references and forward references,
% eg in constructing a table of contents, so Latex may need to be run
% twice to get all the references correct. If you need to run Latex twice
% you may get the warning:
% 
% LaTeX Warning: Label(s) may have changed. Rerun to get cross-refSerences right.


\begin{document}

\title{Coursework for Programming Skills}
\author{Dante Gama Dessavre}
\date{\today}

\makeEPCCtitle

\thispagestyle{empty}

\newpage

\pagenumbering{roman}

\tableofcontents

\newpage

\section*{Acknowledgements}





\newpage
\pagenumbering{arabic}

\section{Introduction}



\subsection{The easy bits}
This is just to show how to break things into sections.



%\subsection{The more difficult bits}
%Some bits are hard.



\section{Planning and Task Division}

You might sometimes want to include equations without numbering them.


\begin{figure}

\begin{center}
\resizebox{0.30\hsize}{!}{\includegraphics{logos/eucrest}}
\end{center}

\caption{The University Crest}
\label{fig:eucrest}

\end{figure}

\section{Implementation, testing and documenting}

\section{Deployment and Performance Analysis}


\begin{figure}
% GNUPLOT: LaTeX picture
\setlength{\unitlength}{0.240900pt}
\ifx\plotpoint\undefined\newsavebox{\plotpoint}\fi
\sbox{\plotpoint}{\rule[-0.200pt]{0.400pt}{0.400pt}}%
\begin{picture}(1500,1200)(0,0)
\font\gnuplot=cmr10 at 10pt
\gnuplot
\sbox{\plotpoint}{\rule[-0.200pt]{0.400pt}{0.400pt}}%
\put(220.0,113.0){\rule[-0.200pt]{292.934pt}{0.400pt}}
\put(220.0,113.0){\rule[-0.200pt]{0.400pt}{245.477pt}}
\put(220.0,113.0){\rule[-0.200pt]{4.818pt}{0.400pt}}
\put(198,113){\makebox(0,0)[r]{$0$}}
\put(1416.0,113.0){\rule[-0.200pt]{4.818pt}{0.400pt}}
\put(220.0,317.0){\rule[-0.200pt]{4.818pt}{0.400pt}}
\put(198,317){\makebox(0,0)[r]{$0.2$}}
\put(1416.0,317.0){\rule[-0.200pt]{4.818pt}{0.400pt}}
\put(220.0,521.0){\rule[-0.200pt]{4.818pt}{0.400pt}}
\put(198,521){\makebox(0,0)[r]{$0.4$}}
\put(1416.0,521.0){\rule[-0.200pt]{4.818pt}{0.400pt}}
\put(220.0,724.0){\rule[-0.200pt]{4.818pt}{0.400pt}}
\put(198,724){\makebox(0,0)[r]{$0.6$}}
\put(1416.0,724.0){\rule[-0.200pt]{4.818pt}{0.400pt}}
\put(220.0,928.0){\rule[-0.200pt]{4.818pt}{0.400pt}}
\put(198,928){\makebox(0,0)[r]{$0.8$}}
\put(1416.0,928.0){\rule[-0.200pt]{4.818pt}{0.400pt}}
\put(220.0,1132.0){\rule[-0.200pt]{4.818pt}{0.400pt}}
\put(198,1132){\makebox(0,0)[r]{$1$}}
\put(1416.0,1132.0){\rule[-0.200pt]{4.818pt}{0.400pt}}
\put(220.0,113.0){\rule[-0.200pt]{0.400pt}{4.818pt}}
\put(220,68){\makebox(0,0){$0$}}
\put(220.0,1112.0){\rule[-0.200pt]{0.400pt}{4.818pt}}
\put(414.0,113.0){\rule[-0.200pt]{0.400pt}{4.818pt}}
\put(414,68){\makebox(0,0){$1$}}
\put(414.0,1112.0){\rule[-0.200pt]{0.400pt}{4.818pt}}
\put(607.0,113.0){\rule[-0.200pt]{0.400pt}{4.818pt}}
\put(607,68){\makebox(0,0){$2$}}
\put(607.0,1112.0){\rule[-0.200pt]{0.400pt}{4.818pt}}
\put(801.0,113.0){\rule[-0.200pt]{0.400pt}{4.818pt}}
\put(801,68){\makebox(0,0){$3$}}
\put(801.0,1112.0){\rule[-0.200pt]{0.400pt}{4.818pt}}
\put(995.0,113.0){\rule[-0.200pt]{0.400pt}{4.818pt}}
\put(995,68){\makebox(0,0){$4$}}
\put(995.0,1112.0){\rule[-0.200pt]{0.400pt}{4.818pt}}
\put(1188.0,113.0){\rule[-0.200pt]{0.400pt}{4.818pt}}
\put(1188,68){\makebox(0,0){$5$}}
\put(1188.0,1112.0){\rule[-0.200pt]{0.400pt}{4.818pt}}
\put(1382.0,113.0){\rule[-0.200pt]{0.400pt}{4.818pt}}
\put(1382,68){\makebox(0,0){$6$}}
\put(1382.0,1112.0){\rule[-0.200pt]{0.400pt}{4.818pt}}
\put(220.0,113.0){\rule[-0.200pt]{292.934pt}{0.400pt}}
\put(1436.0,113.0){\rule[-0.200pt]{0.400pt}{245.477pt}}
\put(220.0,1132.0){\rule[-0.200pt]{292.934pt}{0.400pt}}
\put(45,622){\makebox(0,0){\shortstack{This is\\the\\$y$ axis}}}
\put(828,23){\makebox(0,0){This is the $x$ axis}}
\put(828,1177){\makebox(0,0){This is a plot of $y=sin(x)$}}
\put(220.0,113.0){\rule[-0.200pt]{0.400pt}{245.477pt}}
\sbox{\plotpoint}{\rule[-0.500pt]{1.000pt}{1.000pt}}%
\put(1306,1067){\makebox(0,0)[r]{sin(x)}}
\multiput(1328,1067)(20.756,0.000){4}{\usebox{\plotpoint}}
\put(1394,1067){\usebox{\plotpoint}}
\put(220,113){\usebox{\plotpoint}}
\multiput(220,113)(3.768,20.411){4}{\usebox{\plotpoint}}
\multiput(232,178)(4.132,20.340){3}{\usebox{\plotpoint}}
\multiput(245,242)(3.825,20.400){3}{\usebox{\plotpoint}}
\multiput(257,306)(3.884,20.389){3}{\usebox{\plotpoint}}
\multiput(269,369)(3.944,20.377){3}{\usebox{\plotpoint}}
\multiput(281,431)(4.326,20.300){3}{\usebox{\plotpoint}}
\multiput(294,492)(4.137,20.339){3}{\usebox{\plotpoint}}
\multiput(306,551)(4.276,20.310){3}{\usebox{\plotpoint}}
\multiput(318,608)(4.693,20.218){3}{\usebox{\plotpoint}}
\multiput(331,664)(4.583,20.243){2}{\usebox{\plotpoint}}
\multiput(343,717)(4.754,20.204){3}{\usebox{\plotpoint}}
\multiput(355,768)(5.034,20.136){2}{\usebox{\plotpoint}}
\multiput(367,816)(5.760,19.940){2}{\usebox{\plotpoint}}
\multiput(380,861)(5.579,19.992){3}{\usebox{\plotpoint}}
\put(398.00,923.50){\usebox{\plotpoint}}
\multiput(404,943)(7.049,19.522){2}{\usebox{\plotpoint}}
\multiput(417,979)(7.288,19.434){2}{\usebox{\plotpoint}}
\put(433.18,1021.10){\usebox{\plotpoint}}
\multiput(441,1040)(8.982,18.712){2}{\usebox{\plotpoint}}
\put(460.41,1076.97){\usebox{\plotpoint}}
\put(471.84,1094.28){\usebox{\plotpoint}}
\put(484.84,1110.41){\usebox{\plotpoint}}
\put(500.42,1124.01){\usebox{\plotpoint}}
\multiput(503,1126)(19.159,7.983){0}{\usebox{\plotpoint}}
\put(519.48,1131.37){\usebox{\plotpoint}}
\multiput(527,1132)(20.136,-5.034){0}{\usebox{\plotpoint}}
\put(539.74,1128.60){\usebox{\plotpoint}}
\put(557.04,1117.38){\usebox{\plotpoint}}
\put(570.79,1101.95){\usebox{\plotpoint}}
\put(582.44,1084.80){\usebox{\plotpoint}}
\put(593.09,1066.99){\usebox{\plotpoint}}
\multiput(601,1053)(8.430,-18.967){2}{\usebox{\plotpoint}}
\put(619.18,1010.54){\usebox{\plotpoint}}
\multiput(625,996)(7.413,-19.387){2}{\usebox{\plotpoint}}
\multiput(638,962)(6.403,-19.743){2}{\usebox{\plotpoint}}
\multiput(650,925)(5.830,-19.920){2}{\usebox{\plotpoint}}
\multiput(662,884)(5.461,-20.024){2}{\usebox{\plotpoint}}
\multiput(674,840)(5.533,-20.004){2}{\usebox{\plotpoint}}
\multiput(687,793)(4.937,-20.160){3}{\usebox{\plotpoint}}
\multiput(699,744)(4.667,-20.224){2}{\usebox{\plotpoint}}
\multiput(711,692)(4.858,-20.179){3}{\usebox{\plotpoint}}
\multiput(724,638)(4.276,-20.310){3}{\usebox{\plotpoint}}
\multiput(736,581)(4.205,-20.325){3}{\usebox{\plotpoint}}
\multiput(748,523)(4.070,-20.352){3}{\usebox{\plotpoint}}
\multiput(760,463)(4.326,-20.300){3}{\usebox{\plotpoint}}
\multiput(773,402)(3.884,-20.389){3}{\usebox{\plotpoint}}
\multiput(785,339)(3.884,-20.389){3}{\usebox{\plotpoint}}
\multiput(797,276)(4.070,-20.352){3}{\usebox{\plotpoint}}
\multiput(810,211)(3.825,-20.400){3}{\usebox{\plotpoint}}
\multiput(822,147)(3.607,-20.440){2}{\usebox{\plotpoint}}
\put(828,113){\usebox{\plotpoint}}
\end{picture}
\caption{Simple Gnuplot example}
\label{fig:gnu}
\end{figure}


\section{Conclusions}

This is the place to put your conclusions about your work. You can
split it into different subsections if appropriate. You may want to include
a subsection of future work which could be carried out to continue your
research.

\begin{thebibliography}{100}

\bibitem{ref:lam} L.Lamport. {\em 1986 Latex User's Guide
and Reference Manual.} Addison Wesley. pp242.

\bibitem{ref:bloggs} F.Bloggs. {\em 1993 Latex Users do it
in Environments} Int. Journal of Silly Findings. pp 23-29.

\end{thebibliography}


\end{document}

